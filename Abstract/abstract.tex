
% Thesis Abstract -----------------------------------------------------


%\begin{abstractslong}    %uncommenting this line, gives a different abstract heading
\begin{abstracts}        %this creates the heading for the abstract page

%\setlength{\parindent}{17.62482pt}
%\setlength{\parskip}{0.0pt plus 1.0pt}


In the field of Soft Robotics, viscoelasticity has been proved beneficial for human assistance applications. The human skeletal muscle system, as well as many soft materials commonly used in soft robotic applications, have viscoelastic properties. Viscoelasticity can be modelled using a set of equations known as the Linear Viscoelastic Models (LVMs). This modelling approach has two main limitations: high mathematical complexity and high computational cost. Here, these limitations are addressed in two ways. Firstly, the Piecewise Linearisation method is used to reduce the mathematical complexity of LVMs. Secondly, a modelling approach based on feedforward artificial neural networks (ANNs) is used to reduce the computational cost. The aim of both modelling approaches is to describe the non-linear, strain-dependent, and time-dependent stress response of seven thermoplastic elastomers.

On the one hand, the implementation of the Piecewise Linearisation method yielded the PL-SLS model and the PL-Wiechert model. Both models were successful in predicting the viscoelastic behaviour of the materials, outperforming similar modelling tools documented in the literature. On the other hand, four different architectures of ANN models are developed, categorized in rate-dependent and rate-independent. Results highlight the rate-dependent architecture as the most suitable. The ANN models achieved a similar prediction performance as the PL models. 

The ANN model for the natural rubber material is further validated in a real-time simulation environment, in Simulink. This soft material is found to be the best candidate to imitate the mechanical properties of the human tendon. On the one hand, the performance prediction of the ANN models is adequate for a sine wave strain input, when the strain rate is constant. On the other hand, the response of the ANN model is unstable under variable strain rates. This highlights an important limitation of the training set used for developing the ANN models, which only contains data for three different strain rates. Finally, the three modelling tools developed in this research are a direct improvement to current modelling approaches. Nonetheless, a richer training set is required to improve the ANN models real-time response.

%Reliable modelling tools capable of predicting this viscoelasticity are required. Therefore, the aim of this research is to systematically compare the performance of two main modelling approaches which are model-driven based, and data-driven based. Specifically, the performance of these approaches when being implemented as part of a control system to predict the mechanical behaviour of viscoelastic materials, is of interest. For this, seven different elastomers were studied. On the model-driven side, the piecewise linearisation (PL) method was applied to the Standard Linear Solid (SLS) model and the Wiechert model. On the data-driven side, a feedforward artificial neural network (ANN) was developed. The viscoelastic properties of the selected materials were extracted using the mechanical tests of tensile strength and stress relaxation. The developed models are desired to account for the non-linear, strain-dependent, and time-dependent stress response of the materials, with the aim of being deployed as part of a control system.

%The findings of the research are as follows. On the model-driven side, only one of the two developed models delivered adequate results, the PL-SLS model. The trade-off between complexity and accuracy of this model was characterized. Nonetheless, it was unable to account for time-dependant properties. On the data-driven side, the developed ANN model was able to account for both strain-dependent and time-dependent properties when being assessed in a steady-state environment. However, the validation of the ANN model during real-time predictions was unsatisfactory. The ANN model does not perform well when the strain rate in its input is varying over time. Nonetheless, when the strain rate is fixed, the ANN model is able to follow any type of strain input signal. Lastly, the design of a bio-inspired series-elastic actuator is conceptualized. The intended application for this is to assist the knee joint during walking activities. The experimental protocol designed to test concepts such as co-contraction and variable recruitment is presented as well. The development of the designed actuator was not possible due to time restriction. Nonetheless, all the information regarding the design process is provided in this work.

%The best ANNs candidates were selected based on three main aspects: performance (mse) during train, validation and test, $R^2$ values across all different strain rates, and performance (mse) when presenting the ANNs with unknown experimental data. Subsequently, their real-time prediction was assessed and compared, against the PL-SLS model, in Simulink. In conclusion, this research presents a detailed description on the design, development and implementation of model-driven and data-driven approaches for the prediction of the viscoelastic mechanical behaviour of soft materials, and assess the performance of these approaches when being implemented as part of a control system in a robotic application.

\end{abstracts}
%\end{abstractslong}

% ----------------------------------------------------------------------

%%% Local Variables: 
%%% mode: latex
%%% TeX-master: "../thesis"
%%% End: 
