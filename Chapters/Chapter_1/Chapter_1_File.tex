\chapter{Introduction} \label{ch1:introduction}
\section{Background}

The field of Soft Robotics deals with the implementation of soft and deformable materials in traditional robotic applications. The interest in this field has increased in the past decade. The latter is due to the outstanding progress in manufacturing of soft and smart materials which can be actuated by heat, light, and magnetism. Early applications in Soft Robotics were inspired in nature, by observing that most biological organisms are not rigid \cite{kim2013soft}. This inspiration gave birth to several soft bio-inspired robots.

The design and development of soft robots is is full of challenges, such as:  actuation of soft materials, development and implementation of soft sensors into soft robots, control systems capable of dealing with the nonlinear behaviour of soft materials, and modelling tools able to accurately predict the mechanical behaviour of soft materials in real-time \cite{laschi2016soft,trivedi2008soft}. Nonetheless, the potential benefits are many, such as: soft robots have the potential of being very dexterous due to the ability of modifying their shape depending on the environment; soft robots can manipulate objects of different shapes, sizes; and most importantly soft robots are safe to interact with humans in the event of an unplanned collision \cite{iida2011soft}. These benefits highlight the potential of soft robotic applications for physical human-robot interaction (PHRI) such as, orthoses, surgical tools, and wearable devices. This research aims to contribute to the technological advance of Soft Robotics for human assistance applications.

For a long time, humans have pursued the idea of increasing their strength, stamina and speed through different means. Nowadays, this is a reality due to the technological advances on wearable robotic devices, commonly known as robotic exoskeletons. This wearable device was motivated by military applications where a soldier is required to carry a heavy load in its back for a long period of time, ultimately causing him injuries or early fatigue. The robotic exoskeleton is able to carry a payload and transmit the payload weight to the ground, ideally relieving the wearer from feeling the payload, which allows the wearer to walk greater distances without premature fatigue. The rigid nature of these devices and the big actuators implemented to achieve forces able to enhance humans, impose many limitations, such as, restriction of body movements, interference on subject natural biomechanics and high inertia which impedes the device to follow the subject intentions smoothly, creating a drag feeling \cite{asbeck2014stronger}.

In the field of Soft Robotics, a soft version of the robotic exoskeleton, formally called soft exosuit, is being developed. Soft exosuits are wearable robotic systems meant to be worn in the same way as clothes. In this context, both the actuation and perception systems are attached to a soft structure made of textiles which is strapped to the human body. Despite the many benefits of this technology, available prototypes cannot deliver enough mechanical power to be called an enhancement device. Therefore, these devices are mainly for human assistance applications. This limitation is caused by the challenge of having a soft structure to transmit the forces generated by the actuators that are distal to the assisted joint. Moreover, these forces can create discomfort to the user. Nonetheless, there are areas of the human body which naturally sustain high amounts of forces which are commonly used in the design of soft exosuits \cite{wehner2013lightweight}. Another alternative to address the latter limitation is to deploy the actuators proximal to the assisted joint, closer to the human skeletal muscle system functionality \cite{park2014design,park2011bio}.

In most soft exosuits, the lack of a robust modelling tool for the prediction of the mechanical behaviour of soft materials, makes is difficult to implement a control system. There are two modelling approaches commonly used for this task. One is based on the Linear Viscoelastic Models (LVMs), the other is based on machine learning models. On the one hand, the LVMs describe the mechanical behaviour of a soft material using two basic mechanical components, a spring and a dashpot. Inside this family of models, there are two models that can be expanded as required. In theory, they are capable of describing the most complex soft material. In practice, expanding a mathematical model is translated into solving more differential equations which at the same time translates into higher computational cost. Therefore, modeling tools based on the LVMs have an important trade-off between achievable accuracy and required complexity. In some cases, deploying these models to hardware, i.e. control systems, is prohibitive. 

On the other hand, the modelling approach using machine learning tools is commonly based on Artificial Neural Networks (ANNs). ANNs excel at identifying complex relationship in large datasets. Their development, specially their training process, is time consuming. However, the time required for an ANN to make a prediction is very small once trained. This is the main advantage of ANNs in comparison to model-driven approaches, which have to perform intensive calculations every time new data is presented as input. Plenty of research can be done on both modelling approaches with the aim of developing a robust modelling tool to accurately describe the complex mechanical behaviour of soft materials. The latter has the potential of increasing the adoption of soft materials in robotic applications, as well as creating more efficient wearable systems.

\section{Motivation}

The global percentage of elderly population is constantly rising. The World Health Organization has estimated that the global percentage of people aged 65 or older will triple by 2050, with respect to 2010 \cite{Colombo2012}. Moreover, the amount of elderly people living alone is also increasing. Solely in the United Kingdom, there are 1 million people aged over 65 living on their own \cite{Hill2019}. Although, the social triggers of the mentioned phenomenons are out of the scope of this research, they represent a strong motivation to push forward the research on soft robotic applications for human assistance. Currently, there are viable concepts of assistive exosuits targeted to increase the quality of life of elderly people during activities of daily living, such as walking over ground, ascending stairs, and using a chair. However, the concept of an assistive exosuit is relatively new, the first documented prototype is dated from 2013 \cite{wehner2013lightweight}. The idea behind an exosuit is to translate the proven concept of a robotic exoskeleton, which is a heavy and bulky wearable device aimed to enhance the strength of humans, into a soft, light weight and compliance version aimed to provide assistance to the elderly or disabled people. 

One of the main challenges in this field of research is the modelling of the mechanical behaviour of soft materials. The accuracy of current modelling approaches is proportional to the required computational power. Due to this, the most accurate modelling tools are difficult to be deployed in control systems where the hardware computational power is limited. The necessity for commercial-grade exosuits for human assistance is pushing the development of robust modelling tools capable of predicting the viscoelastic behaviour of soft materials. Therefore, the main motivation of this research is to contribute to the technological development of robust modelling tools. This can translate into more efficient soft actuators, hence soft exosuits, which will improve the quality of life of the elderly and disabled population in the near future.

\newpage

\section{Aims and Objectives}

\subsection{Aims}

The aims of this research are:

\begin{itemize}
    \item To investigate the concept of implementing viscoelasticity, a property found in the human skeletal muscle system, in soft robotic applications for the assistance of the human lower limb.
    \item To develop a reliable modelling tool to be implemented in a bio-inspired soft actuator model for the prediction of the stress response of soft materials in real-time.
\end{itemize}

\subsection{Objectives}

In accordance to the research aims, the following objectives are identified:

\begin{itemize}
    \item Survey the literature to identify: the terminology related to the biomechanics of the human lower limb, and the current soft robotic developments on the field of human assistance.
    \item Investigate the biomechanics of the human lower limb during activities of daily living.
    \item Characterize the viscoelastic properties of suitable soft materials.
    \item Identify the soft materials with the most similar mechanical properties to the human tendon.
    \item Address current limitations on modelling tools for the prediction of the viscoelastic behaviour of soft materials.
    \item Investigate the performance of current modelling tools under simulated real-time conditions.
    \item Design a bio-inspired soft actuator for human-assistance applications.
\end{itemize}

\section{Research Scope}

This research is focused on developing a reliable modelling tool, able to be implemented in a control system, for the prediction of the mechanical behaviour of soft materials. Therefore, the following is included in the scope of the research. 

The biomechanics from the lower limb of the human body during several activities of daily living are characterized. The potential benefits of using a viscoelastic material instead of the traditional metallic spring in series-elastic actuators, is investigated. For this reason, the following soft materials are studied: ethylene polypropylene rubber (EPR \cite{EPRubber2019}), fluorocarbon rubber (FR \cite{FRubber2019}), nitrile rubber (NR \cite{NRubber2019}), natural rubber with polyester (NatPolR \cite{NatPolRubber2019}),  polyethylene  rubber  (PR \cite{PRubber2019}),  silicone  rubber  (SR \cite{SRubber2019}), and natural rubber (NatR \cite{NatRubber2019}). A recent and accurate modelling approach, based on a piecewise linearisation of the Linear Viscoelastic Models (LVMs), is studied. Furthermore, a systematic analysis on the performance of a popular machine learning tool, Artificial Neural Networks (ANNs), is executed. The analysis includes the effect of the number of neurons, output activation function, training algorithm, and combination of inputs/outputs presented to the ANN. Moreover, the performance of the developed ANN models under simulated real-time conditions, is assessed. Finally, the most suitable modelling approach for this application is identified.

\section{Research Contributions}

This research contributes to the field of Soft Robotics for human assistance by investigating the performance of Artificial Neural Networks, as an alternative to the Linear Viscoelastic Models, for the modelling of viscoelasticity in soft materials. Some parts of this thesis are published in peer-reviewed conference papers. The contributions are summarized as follows:

\begin{itemize}
    \item Three modelling tools for the prediction of the non-linear, strain-dependent, and time-dependent stress response of soft materials. The models are: the PL-SLS model, the PL-Wiechert model, and the ANN model.
    \item Identification of design guidelines for the selection of actuator technologies, based on the kinetic and kinematic parameters of the human body, when developing assistive devices.
    \item Incremental improvement of the piecewise linearisation method by proposing a tolerance criteria to define the required complexity of the method depending on the desired prediction accuracy. 
    \item Concept design of a modular clamp mechanism aim to hold a bundle of strings of soft materials in a bio-inspired series-elastic actuator.
\end{itemize}

\section{Thesis Outline}

This thesis is divided and organized in eight chapters, as follows:
\begin{itemize}
    \item {\bf Chapter 1: } This chapter presents the research background, motivation of the researcher, aims and objectives. Moreover, the specific contributions of this research to the respective field of knowledge, and the scope of the thesis are also presented in here.
    \item {\bf Chapter 2:} The second chapter presents an overview on the field of Soft Robotics. The applications related to human assistance are described in terms of the actuation, perception and control system technologies. The recent trend about implementing the functionality of the human skeletal muscle system in soft robotic application is discussed. Due to this, the terminology related to the human biomechanics is included. In addition to this, the challenges of modelling the mechanical behaviour as well as the modelling tools currently used are introduced in this chapter. Finally, the gaps in the body of knowledge are identified and presented.
    \item {\bf Chapter 3:} In this chapter, the process of extracting design guidelines for robotic assistive devices from clinical studies focused on the human gait analysis is presented. Moreover, the characterization of the kinetic and kinematic parameters of the human lower limb during activities of daily living (ADLs), is performed. A total of four main ADLs are investigated from clinical trials: ground level walking, going up/down steps, going up/down a ramp and sitting down/standing up from a chair. The compiled data is processed to obtain a graphical representation and facilitate the comparison and analysis of the parameters variations from subject to subject and activity to activity. The work done in this chapter was substantial enough to produce a conference paper \cite{solis2017characterization}.
    \item {\bf Chapter 4: } In this chapter, the characterization of a selection of seven thermoplastic elastomers is presented. The mechanical tests of tensile strength and stress relaxation are performed to characterize the viscoelastic properties of the materials. The processing of the data and extraction of the materials parameters are described in detail.
    \item {\bf Chapter 5: } This chapter describes the development process of two modelling tools based on the Linear Viscoelastic Models (LVMs). The work in here is inspired by the Piecewise linearisation method, which has been successful in improving the performance of the LVMs. Two modelling tools are developed, PL-Wiechert and the PL-SLS models. The performance of these models is assessed for the description of the non-linear, strain-dependent, and time-dependent stress response of viscoelastic materials. The work done in this chapter was substantial enough to produce a conference paper \cite{solis2018assessment}.
    \item {\bf Chapter 6: } In this chapter, the development of a modelling tool based on artificial neural networks (ANNs), is presented. The research about implementing ANNs in the field of composite materials is very scarce. Nonetheless, ANNs have been proven useful for many function approximation applications. Two hyper-parameters of the developed ANN models are optimized: the number of neurons in the hidden layers, and the selection of inputs. The performance of the developed ANN models is compared against the developed PL models.
    \item {\bf Chapter 7: } This chapter presents the design concept of a bio-inspired series-elastic actuator. The work on this chapter is focused on design due to time limitations. Nonetheless, the proposed design takes the bio-inspiration from the concept of mimicking the human skeletal muscle functionality. Therefore, a comparison analysis between the mechanical properties of the human tendon and the studied soft materials is performed. Also, the prediction capabilities of the ANN models are evaluated in simulated real-time conditions when being deployed as part of a control system. In here, the design of a clamping mechanism is presented, which is aimed to be the interface between the actuator and a load in a series-viscoelastic actuator.
    \item {\bf Chapter 8: } The final chapter is reserved for the summary, conclusions, and future work.
\end{itemize}