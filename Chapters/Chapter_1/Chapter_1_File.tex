\chapter{Introduction} \label{ch1:introduction}
\section{ Background}

The field of Soft Robotics deals with the implementation of soft and deformable materials in traditional robotic applications. The interest in this field has increased in the past decade. The latter is due to the outstanding developments in terms of manufacturing of soft materials, such as elastomers, and the development of new soft materials which can be stimulated, i.e. deformed, by heat, light, and magnetism. 

The coordination action called RoboSoft, supported by the IEEE Robotics and Automation Society (RAS) and the European Commission, has played a very important role in spreading the awareness of the wide number of soft robotic applications. The RoboSoft committee formally defines the field of Soft Robotics as ``Soft robot/devices that can actively interact with the environment and can undergo `large' deformations relying on inherent or structural compliance'' \cite{laschi2016soft}. The categorization of when a robotic application fall into the field of Soft Robotics depends on the material's Young Modulus, a mechanical property which relates the material's deformation with the amount of stress applied to it; which must be between the range of 10$^{2} - $10$^{6}$ Pascals (Pa). The Young's modulus is usually a measure of a solid material stiffness; a high value refers to a stiff material in the same way as a low value refers to an elastic or soft material. In the context of Soft Robotics, the term compliance is commonly used instead of stiffness since it refers to the adaptability of the material under certain circumstances.

Early applications in Soft Robotics were inspired in nature, by observing that most biological organisms are not rigid, e.g. the human skeleton only contributes with 11\% of an adult's weight, on contrast, the skeletal muscle in our body only contributes with 42\% of the weight \cite{kim2013soft}. This inspiration gave birth to several soft bio-inspired robots, as well as the interest in studying the embodied intelligence of biological organisms. The latter refers to the ability of biological organisms to adapt to different situations by exploding their body morphology and properties. This is one of the main differences between soft and rigid robots. The embodied intelligence of soft robots releases the controller from the task of accurately controlling the position of the robot and of constantly monitoring the working environment; allowing the controller to focus on the execution of commands. This is only possible with the implementation of soft and deformable materials able to automatically adapt to perturbations from the environment, such as uneven terrains and obstacles. Bio-inspired soft robots are now being developed for a broad range of applications, such as locomotion, manipulation, and even replicating biological processes such as digestion. The research done in the field of Soft Robotics has an interesting multidisciplinary potential. For example, the locomotion of caterpillars and snake could be study by building a soft robot which replicates this motion. The knowledge extracted from this could be useful for the development of actuated bendable soft cylinders which ultimately could replace the current rigid tools being used in laparoscopic surgery \cite{rus2015design}.

The mechanical behaviour of soft materials is very difficult to model using traditional mathematical models due to their non-linear, time-dependent and strain-rate-dependent stress response. This great challenge motivated the research in Soft Robotics to develop bio-inspired soft bodies, arms and legs able to perform the task at hand using minimal control. This caused a shift in the traditional design approach for rigid robots from ``rigidity by design, safety by sensors and control'' to the design approach used in soft robots, ``safety by design, performance by control''. The added feature of safety, inherent in most soft robots, allowed the development of Physical Human-Robot Interaction (PHRI) applications \cite{filippini2008toward}. Many other challenges faced by this emerging field are: actuation of soft materials, development and implementation of soft sensors into soft robots, control systems able to deal with the nonlinear behaviour of soft materials, and modelling tools able to accurately predict the mechanical behaviour of soft materials in real-time \cite{laschi2016soft,trivedi2008soft}. Most of these limitations come from the simple fact that soft robots cannot be considered as a chain of rigid links able to rotate or slide as common robots are, but soft robots are deformable and continuous which means that all the foundation in which Robotics is based on, is not easily transferable to the Soft Robotics field. 

The design and development of soft robots is very challenging, but the potential benefits are many such as: soft robots have the potential of being very dexterous due to the ability of modifying their shape depending on the environment; soft robots can manipulate objects of different shapes, sizes; and most importantly soft robots are safe to interact with humans in the event of an unplanned collision \cite{iida2011soft}. These benefits highlight the potential of soft robotic applications for PHRI such as, orthoses, surgical tools, and wearable devices. This research aims to contribute to the technological advance of Soft Robotics for human assistance applications.

For a long time, humans have pursued the idea of increasing their strength, stamina and speed through different means. Nowadays, this is a reality due to the technological advances on wearable robotic devices, commonly known as robotic exoskeletons. This wearable device was motivated by military applications where a soldier is required to carry a heavy load in its back for a long period of time, ultimately causing him injuries or early fatigue. The robotic exoskeleton is able to carry a payload and transmit the payload weight to the ground, ideally relieving the wearer from feeling the payload, which allows the wearer to walk greater distances without premature fatigue. The rigid nature of these devices and the big actuators implemented to achieve forces able to enhance humans, impose many limitations, such as, restriction of body movements, interference on subject natural biomechanics and high inertia which impedes the device to follow the subject intentions smoothly, creating a drag feeling \cite{asbeck2014stronger}.

In the field of Soft Robotics, research is being done to develop a soft version of a robotic exoskeleton, formally called soft exosuit, in an attempt to solve the previously mentioned limitations. Soft exosuits are wearable robotic systems meant to be worn in the same way as clothes, by attaching both the actuation and perception systems into a wearable structure, made of textiles, strapped to the human body which will ultimately assist the wearer's motions. Despite the many benefits of this technology, in comparison to the robotic exoskeleton, current developments of soft exosuits are not able to deliver enough mechanical power to be called an enhancement device. Currently, soft exosuits are considered an assistive device. This limitation is inherent from the idea of having a soft wearable device. The accurate delivery of assistive forces generated by the actuators attached to a wearable textile, or directly to the user's body, is very challenging. In a rigid exoskeleton the reaction forces produced by the actuators are sustained by the exoskeleton mechanical frame, but this is not the case in soft exosuits, where the reaction forces must be dissipated through the worn textile attached to the human body, which generates uncomfortable frictions between the user skin and the textile. Nevertheless, the human body naturally possesses areas able to sustain high amounts of forces, these areas are currently being exploited into soft exosuit designs to prevent discomfort, skin injury, and to increase the efficiency of transmitted forces. The latter principle is implemented in \cite{wehner2013lightweight} where a lower limb soft exosuit using pneumatic muscles is developed. The other, most common approach is to fix a wearable soft material to the skin, as in \cite{park2014design,park2011bio} where pneumatic artificial muscles (PAM) were attached to a soft cloth and disposed in such a way that they can mimic the biological musculoskeletal behaviour of a human foot. Among the available soft exosuit developments, few of them implements a closed loop control system, due to the high complexity of dealing with the non-linearity of soft materials. 

The progress in the field of Soft Robotics has been very quick and very diverse. Most of the research is focused on studying the benefits of using a specific new soft material in a robotic application, leaving the control system on the background. This is mostly due to the complexity of developing a modelling tool to accurately predict the mechanical behaviour of soft materials. As previously mentioned, this is one of the main challenges in Soft Robotics applications and the main focus of this research.

In the literature, most of the works are based on one of two approaches, either a model-driven approach or a data-driven approach. On one hand, the model-driven approach, as suggested, is based on using mathematical models to predict the mechanical behaviour of soft materials. Due to the time-dependent properties, or viscoelasticity, exhibited in most soft materials, model-driven approaches use a set of equations for model viscoelasticity known as the Linear Viscoelastic Models (LVM). These models describe a material using two basic mechanical components, a spring and a dashpot, arranged in different configurations. Inside this group, there are two models which can be expanded as required. In theory, this means these two models could be able to describe the most complex soft material as long as they are expanded as required. In practice, expanding a mathematical model is translated into solving more equations at a given time, and this is translated into demanding more computational power. Therefore, the model-driven approach has a well-known trade-off between achievable accuracy and required computational power, making the deployment of control systems based on this approach prohibitive. 

On the other hand, the data-driven approach is based on machine learning tools, specifically Artificial Neural Networks (ANNs). The implementation of these tools for the prediction of the mechanical behaviour of viscoelastic materials has been researched for more than a decade, showing very good performance. ANNs are very good at identifying patterns inside complex data. The training process is known to take plenty of time, depending on the complexity and amount of data. However, once the ANN is trained, they are able to receive an input and produce an output almost immediately. This is the main advantage of ANNs in comparison to model-driven approaches, which have to perform intensive calculations every time new data is presented as input.

As the field of machine learning advances, better tools are becoming available, which leaves room for performing optimizations on previous works and on assessing new machine learning algorithms. At the time of starting this research and at the best of my knowledge, there was no documented work assessing the performance of ANNs as an alternative to model-driven approach when being deployed as part of a control system for a robotic application. In other words, the real-time prediction performance of neural network has not been assessed. Therefore, this research aims to fill this gap in the body of knowledge.

\section{Motivation}

The global percentage of elderly population is constantly rising. The World Health Organization has estimated that the global percentage of people aged 65 or older will triple by 2050, with respect to 2010 \cite{Colombo2012}. Moreover, the amount of elderly people living alone is also increasing. Solely in the United Kingdom, there are 1 million people aged over 65 living on their own \cite{Hill2019}. Although, the social triggers of the mentioned phenomenons are out of the scope of this research, they represent a strong motivation to push forward the research on Soft Robotics applications for human assistance. Currently, there are viable concepts of assistive exosuits targeted to increase the quality of life of elderly people during activities of daily living, such as walking over ground, ascending stairs, and using a chair. However, the concept of an assistive exosuit is relatively new, the first documented prototype is dated from 2013 \cite{wehner2013lightweight}. The idea behind an exosuit is to translate the proven concept of a robotic exoskeleton, which is a heavy and bulky wearable device aimed to enhance the strength of humans, into a soft, light weight and compliance version aimed to provide assistance to the elderly or disabled people. 

One of the main challenges in this field of research is the modelling of the mechanical behaviour of soft materials. The accuracy of current modelling approaches is restricted by the computational cost required. This prevent them to be deployed in most micro-controllers, due to the limited computational power available in this devices. Commercially ready exosuits might be possible in a decade time, meanwhile there is plenty of research to be done to translate well developed technologies used in Robotics into the field of Soft Robotics. Summarizing, this research can improve the quality of life of the elderly and disabled population by developing a novel modelling approach which will enable current soft actuation technologies to be more reliable when being implemented in soft exosuits.

\section{Aims and Objectives}

\subsection{Aims}

The aims of this research are:

\begin{itemize}
    \item To investigate the concept of implementing viscoelasticity, a property found in the human skeletal muscle system, in soft robotic applications for the assistance of the human lower limb.
    \item To develop a reliable modelling tool to be implemented in a bio-inspired soft actuator for the prediction of the stress response of soft materials in real-time.
\end{itemize}

\subsection{Objectives}

In accordance to the research aims, the following objectives are identified:

\begin{itemize}
    \item Survey the literature to identify: the terminology related to the biomechanics of the human lower limb, and the current soft robotic developments on the field of human assistance.
    \item Investigate the biomechanics of the human lower limb during activities of daily living.
    \item Characterize the viscoelastic properties of suitable soft materials.
    \item Identify the soft materials with the most similar mechanical properties to the human tendon.
    \item Address the limitations of the modelling tools currently being used for the prediction of the viscoelastic behaviour of soft materials, by optimizing a current modelling tool or by developing a new modelling tool.
    \item Develop a better modelling approach, or optimize current ones, to allow the accurate prediction of the viscoelastic stress response of soft materials.
    \item Assess the modelling tool prediction accuracy in a real-time environment.
    \item Design a bio-inspired soft actuation for human-assistance applications.
\end{itemize}

\section{Scope of the research}

This research is focused on developing a reliable modelling tool, able to be implemented in a control system, for the prediction of the mechanical behaviour of soft materials. Therefore, the following is included in the scope of the research. 

The biomechanics from the lower limb of the human body during several activities of daily living are characterized. The potential benefits of using a viscoelastic material instead of the traditional metallic spring in series-elastic actuators, is investigated. For this reason, the following soft materials are studied: ethylene polypropylene rubber (EPR \cite{EPRubber2019}), fluorocarbon rubber (FR \cite{FRubber2019}), nitrile rubber (NR \cite{NRubber2019}), natural rubber with polyester (NatPolR \cite{NatPolRubber2019}),  polyethylene  rubber  (PR \cite{PRubber2019}),  silicone  rubber  (SR \cite{SRubber2019}), and natural rubber (NatR \cite{NatRubber2019}). A recent and accurate model-drive approach, based on a piecewise linearization of the Linear Viscoelastic Models (LVMs) is studied. Furthermore, a systematic analysis on the performance of a popular data-driven approach, Artificial Neural Networks (ANNs) is executed. The analysis includes the effect of the number of neurons, output activation function, training algorithm, and combination of inputs/outputs presented to the network. Moreover, a comparison analysis on the performance of these approaches when being implemented as part of a bio-inspired series-elastic actuator model, in a control system, is performed. The control system requirements are designed to meet the knee joint torque-speed characteristics during level ground walking. Finally, the most suitable modelling approach for this application is identified.

\section{Contributions of the research}

The research contributes to the field of Soft Robotics for human assistance by investigating the performance of Artificial Neural Networks, as an alternative to the Linear Viscoelastic Models, for the modelling of viscoelasticity in soft materials. Some parts of this thesis are published in peer-reviewed conference papers. The contributions are summarized as follows:

\begin{itemize}
    \item Assessment of relationship between the accuracy and the mathematical complexity, of the Piecewise Linearized Standard Linear Solid model (PL-SLS).
    \item (REPHRASE THIS-COME BACK TO THIS WHEN ADDRESSING EXAMINERS CORRECTIONS: "which is to provide insight on the accuracy versus complexity trade-off of applying the PL method to the Wiechert model") Development of a model-driven modelling approach, based on the piecewise linearization method, applied to one of the most complete LVMs, the Wiechert model.
    \item Development of a data-driven modelling approach, based on ANNs, for the prediction of the mechanical behaviour of soft materials.
    \item (This might need to be removed) Concept design of a bio-inspired series-elastic actuator, which uses the data-driven modelling approach to predict the mechanical behaviour of the soft elastic element in the actuator.
\end{itemize}

\section{Outline of the thesis}

This thesis is divided and organized in six chapters, as follows:
\begin{itemize}
    \item {\bf Chapter 1: } This chapter presents the research background, motivation of the researcher, aims and objectives. Moreover, the specific contributions of this research to the respective field of knowledge, and the scope of the thesis are also presented in here.
    \item {\bf Chapter 2:} The second chapter describes the state of the art on Soft Robotics applied to human assistance. The latter is organized in three main sections: soft actuation technologies, soft perception systems and soft robotic controllers. However, the main focus of this chapter is directed to innovative soft actuation technologies by describing their advantages and limitations. The design approach of mimicking the human musculoskeletal functionality is also described in here.  Finally, this literature review highlighted the research opportunities of implementing a soft material as the elastic element of traditional series-elastic actuators, replacing the commonly used metallic spring. To do this, a reliable modelling tool, able to predict the mechanical behaviour of soft materials, must be developed.
    \item {\bf Chapter 3:} In this chapter, the terminology related to human biomechanics is presented. Also, the mechanical properties of the human muscle-tendon component is discussed. Polymer materials were identified as potential candidate to imitate the mechanical properties of the human tendon. Moreover, the characterization of the kinetic and kinematic parameters of the human lower limb during activities of daily living (ADLs), is performed. A total of four main ADLs are investigated from clinical trials: ground level walking, going up/down steps, going up/down a ramp and sitting down/standing up from a chair. The compiled data is processed to obtain a graphical representation and facilitate the comparison and analysis of the parameters variations from subject to subject and activity to activity. The work done in this chapter was substantial enough to produce a conference paper \cite{solis2017characterization}.
    \item {\bf Chapter 4: } In this chapter, the characterization of a selection of seven composite materials is presented. The mechanical tests of tensile strength and stress relaxation are performed to characterize the viscoelastic properties of the materials. The processing of the data and extraction of the material parameters are described in detail.
    \item {\bf Chapter 5: } (REPHRASE) This chapter describes the investigation performed on one of the most recently developed modelling tool focused on soft materials. In here, two models were developed based on the one available in the literature. These are the PL-Wiechert and the PL-SLS models. The performance of these models is assessed for the description of the nonlinear, time-dependent, and strain-dependent stress response of viscoelastic materials. The work done in this chapter was substantial enough to produce a conference paper \cite{solis2018assessment}.
    \item {\bf Chapter 6: } In this chapter, the development of a data-driven modelling tool, based on artificial neural networks (ANNs), is presented. The research about implementing ANNs in the field of composite materials is very scarce. Nonetheless, ANNs have been proven useful for many function approximation applications. Two hyper-parameters of the developed ANN were optimized: the number of neurons in the hidden layers, and the selection of inputs. The ANN model was able to achieve a better accuracy than the PL-SLS model and also to generalize better, i.e. to account for the velocity-dependent stress response of the materials.
    \item {\bf Chapter 7: } This chapter presents the design concept of a bio-inspired soft actuator. The work on this chapter is focused on design due to time limitations. Nonetheless, the concept design takes the bio-inspiration from the concept of mimicking the human skeletal muscle functionality. Therefore, a comparison analysis between the mechanical properties of the human tendon and the studied soft materials is performed. Also, the ANN model is evaluated in real-time to observe its prediction capabilities when being deployed as part of a control system. In here, the design of a clamping mechanism is presented, which is aimed to be the interface between the actuator and a load in a series-viscoelastic actuator. Finally, the experimental protocol documented as part of this design process is presented in here, which is focused on investigating concepts such as co-contraction and variable recruitment, commonly functionalities of the human muscle-tendon component.
    \item {\bf Chapter 8: } The final chapter is reserved for the summary, conclusions, and future work.
\end{itemize}