% %Decide what to do with this sections
% \section{Bioinspired Solutions}
% \subsection{Series and Parallel Elasticity}
% \subsection{Variable Recruitment}

% Findings in series-parallel-elastic actuators (SPEA). The main author in this research field is  Mathijsseeb G., his oldest contribution in this field is in 2013. He explored the concept of variable recruitment in 2015.

% Wesley Roozing (2018), has continued this work and proved the high energy efficiency of the SPEA.

% \subsection{Multi-motor Actuators}

% Lastly, Tom Verstraten (2018) is researching the concept of multi-motor-actuation. The concept of Bidirectional Antagonist Variable Stiffness was also discussed.

% I did some important reviews on these:
% \begin{itemize}
%     \item Book: Biomechanics of Anthropomorphic Systems. Chapter:  should anthropomorphic systems be redundant?
%     \item Kinematic ally redundant actuators, a solution for conflicting torque-speed requirements
% \end{itemize}

% \subsection{The Soft Elastic Element}
% The contents of this section should be as follows:
% \begin{itemize}
%     \item Background regarding advantages of soft materials, such as intrinsic compliance (optional)
%     \item Importance of viscoelasticity in the human biomechanics
%     \item Background regarding soft materials with viscoelasticity
%     \item Background regarding the implementation of synthetic/soft materials as a suitable replacement of the human tendon. Synthetic materials such as Polyethylene Terephthalate has been used in tendon reconstruction applications. This piece of information briefly justify the selection of composite materials based on rubber and polymers. The reference is: Shalimar Abdullah - Usage of synthetic tendons in tendon reconstruction.
% \end{itemize}

% \section{The Human Musculoskeletal System}

% \subsection{Muscle Redundancy}

% What does redundancy refer to?

% Studies on the ``redundancy" of the human musculoskeletal system and the problematic on how the brain addresses the control of an ``overcomplete" system date back to the 1950s. In mechanical systems, redundancy refers to having more than two linear actuators controlling a single degree of freedom (DOF). In robotics applications, the common choice is to use a single rotational actuator,  or few linear actuators, per degree of freedom (DOF). The hypothesis of the human musculoskeletal system being redundant, i.e. many muscles, only able to contract, controls one DOF, is valid from a strictly mechanical point of view. G. Loeb explains why this hypothesis is biased to the results obtained from restrictive experimentation where subjects are asked to perform ``simple" movements, such as ankle dorsiflexion. These experiments often overlook the understanding of subjects about the task to be performed, which leads to a great variability of muscles activation for the same task depending on the subject intention. The latter means that more muscles than expected are potentially involved in the control of  a single DOF \cite{loeb2000overcomplete}.

% Why are anthropomorphic systems redundant?

% However, G. Loeb disproves the existence of redundancy in the human musculoskeletal system by simple calculating the required number of muscles (two per DOF) to control a human limb with N DOF. In his work, it is mentioned that less muscles than the minimum required amount of 2*N intervene in the motion of a specific human limb, suggesting that the human musculoskeletal system might be incomplete (underactuated) rather than overcomplete. In reality, the human musculoskeletal system may have barely enough muscles to perform the wide range of activities involved in a real-life environment. The latter is strongly supported in the recent work of A. Marjaninejad \etal in where a systematical analysis regarding the redundancy of anthropomorphic systems is performed CITE. 

% What are the benefits of redundancy?