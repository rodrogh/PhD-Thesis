Intro
%On the other hand, several works focus on the characterization and testing of soft technologies as \cite{wehner2012exp} where an accurate characterization of a PAM is performed, which directly contribute to the development of previously mentioned soft exosuits. Many other technologies are being researched as well, such as, shape memory alloys (SMAs), which are metal alloys capable of changing its form when  properly stimulated. One of the main drawbacks of this technology is the amount of energy and time required to heat metal alloy and activate its shape shifting ability. The latter limitation is highlighted in \cite{Stirling2011}, where soft orthoses for foot and knee were developed SMA springs embedded in a soft fabric. The response of the device was to slow to be used for human assistance. Another commonly implemented actuation technology are cable-driven actuators, typically consisting of an electric motor and a Bowden cable. This electromechanical system create pulling forces by applying tension to a cable fixed in two different points. The cables must be routed along the body of the wearer to deliver the acting force to the joint of interest \cite{asbeck2013biologically,asbeck2015biologically}. The stroke length is not limited in this type of technology, as is the case for PAMs, since Bowden cables are flexible and can be routed in many different ways.

%Many soft sensing technologies have also been developed, being hyper-elastic strain sensors the most common one, which are based on the change in resistance of a liquid metal alloy inside conducts embedded into an elastic polymer which allows the measurement of the body limb position and sustained forces; its performance allowed the development of a soft artificial skin \cite{park2012design}. This technology was later implemented in a soft exosuit to allow for the measurement of the human biomechanics. A comparison analysis between the latter approach and the well-established motion capture technology was performed in \cite{mengucc2014wearable}, where the implementation of strain sensors to capture the human biomechanics showed remarkable results.


Fourthly, a comparison analysis between the mechanical properties of the studies soft materials and the human tendons involved in the knee motion is presented. Earlier it was mentioned the decision of focusing on the knee joint of the human lower limb, due to the large participation it has on the activities of daily living (ADLs). Finally, the last section of this chapter presents the summary of the relevant findings.


\section{Comparison Analysis}




{\huge The Best Candidate Explanation}

At this stage, the elastic and viscoelastic properties of the soft materials and the human tendon have been compared against each other. The results highlights that the Flourocarbon Rubber  has the most similarities to the human tendon mechanical properties. This is based on the two tables summarizing the elastic and viscoelastic properties of both.
To-do List
\begin{itemize}
    \item Define the criteria to choose which material is the most suitable
    \item Elastic parameters
    \begin{itemize}
        \item Stiffness at Toe region (small deformation)
        \item Stiffness at Elastic region (long deformation)
        \item Ultimate load
        \item Ultimate elongation
        \item Matching Factor(increment factor of the material cross-sectional area required to match the human tendon strength)
        
    \end{itemize}
    \item Viscoelastic parameters
    \begin{itemize}
        \item 
    \end{itemize}
\end{itemize}
