Fourthly, a comparison analysis between the mechanical properties of the studies soft materials and the human tendons involved in the knee motion is presented. Earlier it was mentioned the decision of focusing on the knee joint of the human lower limb, due to the large participation it has on the activities of daily living (ADLs). Finally, the last section of this chapter presents the summary of the relevant findings.


\section{Comparison Analysis}




{\huge The Best Candidate Explanation}

At this stage, the elastic and viscoelastic properties of the soft materials and the human tendon have been compared against each other. The results highlights that the Flourocarbon Rubber  has the most similarities to the human tendon mechanical properties. This is based on the two tables summarizing the elastic and viscoelastic properties of both.
To-do List
\begin{itemize}
    \item Define the criteria to choose which material is the most suitable
    \item Elastic parameters
    \begin{itemize}
        \item Stiffness at Toe region (small deformation)
        \item Stiffness at Elastic region (long deformation)
        \item Ultimate load
        \item Ultimate elongation
        \item Matching Factor(increment factor of the material cross-sectional area required to match the human tendon strength)
        
    \end{itemize}
    \item Viscoelastic parameters
    \begin{itemize}
        \item 
    \end{itemize}
\end{itemize}
