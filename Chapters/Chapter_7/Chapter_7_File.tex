\chapter{Summary, Conclusions, and Future Work}

\section{Summary}

In \Cref{ch1:introduction}, the aims and objectives of the research were outlined. The aims of the research were to investigate the concept of implementing viscoelasticity, a property found in the human skeletal muscle system, in soft robotics applications for the assistance of the human lower limb. Additionally, this research aims to address current limitations on the modelling of the viscoelastic mechanical behaviour of soft materials, by developing a reliable modelling tool. Moreover, to investigate recent alternative modelling approaches for the prediction of the viscoelastic mechanical properties of soft materials; and to assess the performance of the selected modelling tools candidates when being implemented in a real-time environment, is of interest. Lastly, this research aims to design a bio-inspired soft actuator which benefits from the viscoelastic properties of soft materials. In this section, the objectives of the research are evaluated in terms of the achieved results, as follows:
\\[1em]
\noindent \textbf{ \large{ Survey the literature to identify: the terminology related to the biomechanics of the human lower limb, and the current soft robotic developments for the field of human assistance.}}

The tendency of mimicking the human muscle skeletal system functionality when developing soft robotic applications for human assistance, was presented in \Cref{ch2:Literature}. Furthermore, the most matured soft actuation technologies commonly used were identified as: Pneumatic/hydraulic artificial muscles, and cable-driven actuators. The benefits and limitations of these technologies were described. In the field of soft sensing applications, the literature is mainly focused on elastomers with embedded micro-conductive channels. The lack of literature describing the modelling of the soft materials used for these applications was detected as a research opportunity. Lastly, the implementation of soft actuation technologies as a way to imitate the functionality of the muscle-tendon component is mainly focus on the contractile element rather than the elastic element. From here, the idea of researching the potential of using soft material to replicate the human tendon functionality was proposed and addressed in the next chapter.
\\[1em]
\noindent \textbf{\large{  Investigate the biomechanics of the human lower limb during activities of daily living.}}

In \Cref{sec:mimicHSMS}, the terminology on the biomechanics of the human lower limb during activities of daily living was presented. Special attention is given to the mechanical model commonly used in soft robotic applications, the Hill's model, when mimicking the functionality of the human skeletal muscle system. Moreover, the characterization of the kinematic and kinetic parameters of the hip, knee, and ankle joints was performed. The potential of using the compiled data was identified and a conference paper was produced. The latter paper presents several visualization techniques which can be useful for the design phase of any soft robotic application for human assistance of the lower limb. The complete collected dataset is presented, visually, in \Cref{appendixA}. Polymer materials were identified as potential candidates to be used as part of a bio-inspired actuator due to the similarities of these materials with the mechanical properties of the human tendon, and due to the medical application in which it has been used, e.g. tendon reconstruction. Lastly, viscoelasticity, a property of most soft materials, is now being research as an alternative to improve mature actuation technologies, such as series-elastic actuators. The concept of a series-viscoelastic actuator is very recent and further validates the feasibility of this research.
\\[1em]
\noindent \textbf{\large{ Characterize the viscoelasticity of suitable soft materials.}}

The characterization process of seven soft materials from the family of composite materials is presented in \Cref{ch:characterizationSoft}. The studied rubber-based materials are as follows: Polyethylene Rubber (PR), Ethylene Polypropylene Rubber (EPR), Natural Rubber with Polyester (NatPolR), Natural Rubber (NatR), Silicone Rubber (SR), Fluorocarbon Rubber (FR), and Nitrile Rubber (NR). The concept of elasticity and viscoelasticity were described in here. The mechanical tests of tensile strength and stress relaxation were performed. The processing of the collected data was described in detail. Moreover, particular attention was paid to find the elastic region of the studied materials since these regions dictates the safest working conditions for each material. This information is latter used for the design of a bio-inspired soft actuator.
\\[1em]
\noindent \textbf{\large{ Identify the soft materials with the most similar mechanical properties to the human tendon.}}

The motivation behind characterizing many composite materials is to identify the material with the most similar viscoelastic properties as the human tendon. This is another step towards implementing viscoelasticity to soft robotic applications for human assistance. The process to identify the best candidate was presented in \Cref{ch7:ModelingBio}, as part of the design process of a bio-inspired actuator. In this comparison, the human tendons involved in the motion of the knee joint were investigated. These are the patellar and quadriceps tendons. The results showed a large difference between the elastic properties of the materials and properties of the human tendon. Nonetheless, their viscoelastic behaviour was found to have similarities. Due to this, the calculation of a matching factor was performed. The idea is to combine several pieces of a material into a bundle to increase the effective cross-sectional area of the material, hence its elastic properties. Lastly, the matching factors resulting from the latter were reported from which the best candidate soft material to match the human tendon properties was found to be the natural rubber with polyester (NatPolR) material.
\\[1em]
\noindent \textbf{\large{ Address the limitations of the modelling tools currently being used for the prediction of the viscoelastic behaviour of soft materials, by developing new modelling tool or optimizing a current one. }}

In \Cref{sec:ChapterModellingLVM} the current modelling approach for the prediction of stress response of soft materials is presented. In general, modelling approaches are based on the Linear viscoelastic Models (LVMs), which are able to describe any viscoelastic materials using a combination of two basic components: a spring and a damped. Nonetheless, the accuracy of this models is tightly linked to a high complexity in terms of computational cost. One of the recent methods described to circumvent the latter limitation was found in the literature as the Standard Linear Solid with Strain-Dependent Stiffness model (Std. Lin. SSD). This model is able to account for the nonlinear stress response of viscoelastic materials by implementing a piecwise linearization method on the equilibrium spring of the LVM used. Nonetheless, the model has not been validated when predicting the velocity-dependent properties of viscoelastic materials. Therefore, a case study is performed on this model, aimed to address its limitations. The latter is done by applying the piecewise linearization (PL) method to a more complex LVM, the Wiechert model. The hypothesis behind this is that a more complex model might be able to account for velocity-dependencies of the viscoelastic materials. During this process, the Std. Lin. SSd was optimized in many aspects that a differentiation is made between the version found in the literature and the one developed in here, called the PL-SLS. One of the main contributions of the latter was proposing a tolerance criteria which related the complexity of the model with its accuracy. The developed PL-Wiechert model was incompatible with the PL method and only the PL-SLS model is further analysed. The work done in here was substantial enough to produce a second conference paper. Lastly, the generalization of the PL-SLS was found to be poor in most cases, i.e. the model was not able to accurately account for velocity-dependencies. Hence, an alternative modelling approach was investigated in the following chapter.
\\[1em]
\noindent \textbf{\large{ Assess the modelling tool prediction accuracy in a real-time environment}}

Prior to assess the model performance in a real-time environment, a different modelling approach was investigated and is presented in \Cref{ch6:ANN}. A data-driven approach was proposed, based on feedforward artificial neural network.  The literature available on the implementation of artificial neural networks for the prediction of the stress response of viscoelastic materials is very scarce. Nonetheless, this research area is rapidly growing due to the proven potential of ANN to approximate complex functions. The data from the same seven materials was adapted and used for the training of the artificial neural networks (ANN). The developed ANN model was aimed to address the limitations of the PL-SLS model, which is not accounting for velocity-dependencies. The ANN model achieved better accuracy and better generalization capabilities, than the PL-SLS model. The latter was assessed by statistical parameters such as the root mean square error and coefficient of determination. At first glance, this suggested that the ANN model was able to account for velocity-dependencies. Nonetheless, the real-time validation performed in Simulink showed a completely different result. This was presented in \Cref{ch7:ModelingBio}. In summary, the ANN model was unstable due to the varying strain rate used for testing. For a constant strain rate, as the one used when validating the accuracy, the ANN model performed as expected. The PL-SLS model was not validated in here because its formulation requires past values of the stress response which are only obtainable by experimentation. 
\\[1em]
\noindent \textbf{\large{ Design a bio-inspired soft actuation for human-assistance applications}}

The design of a series-viscoelastic actuator is presented in \Cref{ch7:ModelingBio}. due to time restrictions, the information presented in here is focused on the concept design. The proposed actuator is aimed to assist the knee joint, hence the torque knee requirements of the joint during walking activities were used as guidelines for the selection of the electric motor and gearbox combination. Moreover, and in line with the aim of mimicking the human skeletal muscle functionality, a experimental protocol is designed to investigate concepts like co-contraction and variable recruitment, two phenomena found in human muscles. Lastly, the design of a clamping device to be used in the experimentation was designed. The clamping mechanism was designed to be detachable and allow the testing of many materials without modifying the experimental setup.

\section{Conclusions}

In conclusion, this research presented two novel modelling approaches, one model-driven and the other data-driven, for the prediction of the viscoelastic properties of soft materials. Specifically, the developed model were aimed to account for the nonlinear, velocity-dependent, and strain-dependent stress response of the soft materials. In the model-driven approach, two mathematical models were developed and assessed: the PL-Wiechert model and the PL-SLS model. The former one was incompatible with the picewise linearization method proposed in the literature. In contrast, the PL-SLS model showed adequate accuracy achieving a lowest value of 5\% of normalized root mean square error between predicted and experimental values. The generalization capabilities of the PL-SLS model showed varied results which suggested that the PL-SLS model performance is biased towards soft materials with dominant elastic properties, i.e. small velocity-dependencies. The latter is reflected in the generalization errors reported in \Cref{tbl:ANNvsSLS}. This limitation motivated the investigation of an alternative modelling tool which was based on a feedforward artificial neural network. In this case, the selection of input parameters, and the number of neurons in the hidden layer, were optimized. From there it followed that the topology code-named ff4 had the best generalization error and was hence chosen as the best candidate. Nonetheless, the ff4 topology made use of the strain and strain rate as its inputs. This initially, allowed the ANN model to achieve better accuracy, in comparison to the PL-SLS model (\Cref{tbl:ANNvsSLS}), but then it was found detrimental for the performance of the ANN model in real-time. The latter is directly related to the limited number of strain rates used for training, but also to the varying strain rate used during the test. The real-time analysis was performed in Simulink, in here the test ANN model also delivered a negative offset when the values of its input were both zero. All these findings suggest that having the strain rate as input to the network require a more comprehensive dataset to allow the ANN model to learn the relationship between the stress and the strain-rate. Alternatively, another topology could be used, which is not rate-dependent, for real-time predictions. In this case, stability during real-time prediction would be favour over excellent large prediction accuracy.

The investigation performed on the concept of mimicking the human skeletal system by implementing viscoelastic materials yielded three important conclusions. Firstly, research is being done on series-elastic actuators to convert them into series-viscoelastic actuators, due to the proven benefits of adding viscoelasticity to robotic applications for human assistance. The latter highlights the feasibility of the research presented in here. Secondly, the matching factor analysis performed in \Cref{ch7:ModelingBio} identified the best soft materials to match the viscoelastic properties of the human tendon, which can be used as design guidelines. Thirdly, a growing interest of mimicking the functionality of the human muscles such as co-contraction and variable recruitment are already being applied to robotic applications. However, this has the potential to be applied to soft robotic applications. Due to this, a experimental protocol was designed to investigate the latter but unfortunately it was not possible to execute in time.

Finally, this research has further proven the potential of ANN model for the prediction of the viscoelastic properties of soft materials. Nonetheless, this is another step taken towards developing a reliable and easy to deploy model tool, which will ultimately translate into the optimization and development of new soft robotic applications.

\section{Future Work}

The findings of this research have allowed the identification of the following issues still needed to be addressed:

\begin{itemize}
    \item Dynamic mechanical analysis can be used to characterize the soft materials under a wide range of frequencies to create a more comprehensive dataset to train ANN models.
    \item Further investigation on the concept of series and parallel elasticity in combination with soft materials.
    \item Validation of the ANN model when being deployed as part of a control system must be investigated.
    \item Further characterization of the viscoelasticity of soft materials could be done by extracting their hysteresis, creep and cyclic behaviour.
    \item Different input combinations for the ANN model could be explored, mainly the ones which do not have the strain rate of the material as input.
    \item Alternative types of ANNs could be investigated, such as Radial Basis Neural Networks and Recurrent Artificial Neural Networks.
    \item Further investigation on the implementation of the concepts of co-contraction and variable recruitment could be performed for soft robotics applications.
    \item Further investigation on the field of the human skeletal system functionality, such as muscle redundancy, which is a compatible approach for cable-driven actuators.
\end{itemize}