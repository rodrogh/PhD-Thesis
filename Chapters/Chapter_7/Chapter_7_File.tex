\chapter{Summary, Conclusions, and Future Work}

\section{Summary}

The aim of this research, is to investigate the concept of implementing viscoelasticity, a property found in the human skeletal muscle system, in soft robotics applications for the assistance of the human lower limb. Narrowing this a little further, special focus is put into the human knee joint and the concept of a series-viscoelastic actuator. In addition to this, this research aims to address current limitations on the modelling of soft materials. Specially the nonlinear, time dependent, and strain dependent stress response of soft materials is of interest. The latter is systematically investigated by developing three modelling tools. Two of them are based on the Linear Viscoelastic Models, whereas the third one is based on artificial neural networks (ANNs). The motivation behind investigating different modelling approaches is to address the known limitations of traditional modelling tools, based on mathematical models, when being deployed as part of a control system. These limitations are high complexity and large computational power requirements. Due to the latter, the performance of the developed ANN model is assessed in a simulated real-time environment using Simulink. The simulation is based on the simplified model of a bio-inspired series-viscoelastic actuator (SVA). The main difference between a SVA and traditional series-elastic actuators is in the type of elastic element implemented. In here, the aim is to implement a bundle of strings made of a specific soft material. Lastly, this last chapter of the thesis provides a summary of the actions taken towards completing the aims and objectives presented in \Cref{ch1:introduction}, as follows: 
\\[1em]
\noindent \textbf{ \large{ Survey the literature to identify: the terminology related to the biomechanics of the human lower limb, and the current soft robotic developments for the field of human assistance.}}

The tendency of mimicking the human muscle skeletal system functionality when developing soft robotic applications for human assistance, is presented in \Cref{ch2:Literature}. Furthermore, the most matured soft actuation technologies commonly used are identified as: pneumatic/hydraulic artificial muscles, and cable-driven actuators. The benefits and limitations of these technologies are described. In the field of soft sensing applications, the literature is mainly focused on elastomers with embedded micro-conductive channels. The lack of literature describing the modelling of the soft materials used for these applications is detected as a research opportunity. Lastly, the implementation of soft actuation technologies as a way to imitate the functionality of the muscle-tendon component is mainly focused on the contractile element rather than the elastic element. Polymer materials are identified as potential candidates to be used as part of a bio-inspired actuator due to the similarities of these materials with the mechanical properties of the human tendon, and due to the medical application in which they have been used, e.g. tendon reconstruction. Lastly, viscoelasticity, a property of most soft materials, is now being researched as an alternative to improve mature actuation technologies, such as series-elastic actuators. The concept of a series-viscoelastic actuator is very recent and further validates the feasibility of this research.
\\[1em]
\noindent \textbf{\large{  Investigate the biomechanics of the human lower limb during activities of daily living.}}

In \Cref{sec:chapterDesignGuidelines}, kinetic and kinematic parameters of the human lower limb during activities of daily living are presented. Special attention is given to the mechanical model commonly used in soft robotic applications, the Hill's model, when mimicking the functionality of the human skeletal muscle system. The potential of using the compiled data is identified and a conference paper is produced from the work performed in this chapter. The latter paper presents several visualization techniques which can be useful for the design phase of any soft robotic application for human assistance of the lower limb. The complete collected dataset is presented in \Cref{appendixA} in the form of charts.
\\[1em]
\noindent \textbf{\large{ Characterize the viscoelastic properties of suitable soft materials.}}

The characterization process of seven soft materials from the family of composite materials is presented in \Cref{ch:characterizationSoft}. The studied rubber-based materials are as follows: Polyethylene Rubber (PR), Ethylene Polypropylene Rubber (EPR), Natural Rubber with Polyester (NatPolR), Natural Rubber (NatR), Silicone Rubber (SR), Fluorocarbon Rubber (FR), and Nitrile Rubber (NR). The concept of elasticity and viscoelasticity are described in here. The mechanical tests of tensile strength and stress relaxation are performed. The processing of the collected data is described in detail. Moreover, particular attention is paid to find the elastic region of the studied materials since these regions dictates the safest working conditions for each material. This information is latter used for the design of a bio-inspired soft actuator.
\\[1em]
\noindent \textbf{\large{ Identify the soft materials with the most similar mechanical properties to the human tendon.}}

The motivation behind characterizing many composite materials is to identify the material with the most similar viscoelastic properties as the human tendon. This is another step towards implementing viscoelasticity in soft robotic applications for human assistance. The process to identify the best candidate is presented in \Cref{ch7:ModelingBio}, as part of the design process of a bio-inspired actuator. In this comparison, the human tendons involved in the motion of the knee joint are investigated. These are the patellar and quadriceps tendons. The results showed a large difference between the elastic properties of the materials and properties of the human tendon. Nonetheless, their viscoelastic behaviour is found to have similarities. Due to this, the calculation of a matching factor is performed. The idea is to combine several pieces of a material into a bundle to increase the effective cross-sectional area of the material, hence its elastic properties. Lastly, the soft material with the most similar properties as the human tendon, in terms of the achievable stiffness, is identified by looking at the obtained matching factors. This material is the Natural Rubber with Polyester material (NatPolR).
\\[1em]
\noindent \textbf{\large{ Address current limitations on modelling tools for the prediction of the viscoelastic behaviour of soft materials. }}

In \Cref{sec:ChapterModellingLVM} the current modelling approach for the prediction of stress response of soft materials is presented. In general, modelling approaches are based on the Linear viscoelastic Models (LVMs), which can describe any viscoelastic materials using a combination of two basic components: a spring and a dashpot. Nonetheless, the accuracy of these models is tightly linked to a high complexity and large computational power requirements. One of the recent methods described to circumvent the latter limitation is found in the literature as the Standard Linear Solid model with Strain-Dependent Stiffness (Std. Lin. SSD). This model can account for the nonlinear stress response of viscoelastic materials by implementing a piecewise linear regression on the equilibrium spring stiffness $k_e$ found in the Standard Linear Solid model (SLS). Nonetheless, the model has not been validated when predicting the velocity-dependent properties of viscoelastic materials. Therefore, a case study is performed on this model, aimed to address its limitations. The latter is done by linearising the Wiechert model, a more complex model from the family of LVMs, using the Piecewise Linearisation method. The hypothesis behind this is that a more complex model might be capable of accounting for the time dependent stress response of the viscoelastic materials. The Piecewise Linearisation method implemented in this work is optimized in many aspects (\Cref{fig:FCFittingProcess}). Hence, the resulting PL-SLS and PL-Wiechert models are different, and a direct improvement, from the Std. Lin. SSD. The proposed tolerance criteria, which takes the variation on the slope of the stress-strain curve to define the number of strain segments to collocate, provided crucial information about the benefits and limitations of the PL method. The models developed in here can account for the nonlinear, time dependent, and strain dependent stress response of soft materials, and in fact, outperformed the Std. Lin. SDS model. The work done in here is substantial enough to produce a second conference paper. In addition to this, a different modelling approach is investigated in \Cref{ch6:ANN}. This model is based on machine learning algorithms, specifically on feedforward artificial neural networks (ANNs). The implementation of an ANN model for the prediction of the stress response of viscoelastic materials is poorly documented. Nonetheless, this area of research is rapidly growing due to the proven potential of ANNs to approximate complex functions. The data from the same seven materials is used for the training of the ANN models. The objective is to overcome the limitations of modelling tools based on the LVMs. These limitations are high complexity and computational cost. The performance of the ANN models is very similar to the PL methods. The latter is assessed by statistical parameters such as the root mean square error and the coefficient of determination. In summary, the three developed models outperforms the Std. Lin. SDS in the task of predicting the nonlinear, time dependent, and strain dependent stress response of soft materials.
\\[1em]
\noindent \textbf{\large{ Investigate the performance of current modelling tools under simulated real-time conditions.}}

The developed ANN model is chosen for further validation. In this case, the PL methods are not suitable due to their formulation which make use of previous values in time of the stress response of the material. This contradicts the main motivation behind using a viscoelastic element in series with a load. The real-time validation performed in Simulink highlighted the limitations of the developed ANN model, presented in \Cref{ch7:ModelingBio}. In summary, the ANN model is unstable due to the varying strain rate used for testing. For a constant strain rate, as the one used when validating the accuracy, the ANN model performed as expected.
\\[1em]
\noindent \textbf{\large{ Design a bio-inspired soft actuation for human-assistance applications}}

The design concept of a series-viscoelastic actuator is presented in \Cref{ch7:ModelingBio}. The proposed actuator is aimed to assist the knee joint, hence the torque knee requirements of the joint during walking activities are used as guidelines for the selection of the electric motor and gearbox combination. Lastly, the design of a clamping device to be used in the experimentation is designed. The clamping mechanism was designed to be detachable and allow the testing of many materials without modifying the experimental setup.

\section{Conclusions}

In conclusion, in this research the development of three modelling tools is presented. They are the PL-SLS model, the PL-Wichert model, and the ANN model. All three of these models are successful in describing the nonlinear, time dependent, and strain dependent stress response of the studied soft materials. The accuracy of all models when predicting the stress-strain curve of the materials is better than the accuracy reported for the Std. Lin. SDS model. On the one hand, the Piecewise linearisation method implemented in here is optimized in many ways (\Cref{fig:FCFittingProcess}). Due to this, the models developed in here are differentiated from the Std. Lin. SDS and named as PL-SLS model and PL-Wiechert model. On the other hand, the ANN model, based on feedforward artificial neural networks also outperformed the Std. Lin. SDS. The optimization performed to the ANN models indicated that the architecture FFRD4 is the most suitable for the prediction of viscoelasticity. The inputs of this architecture are the strain and the strain rate. The output is the stress. Due to the addition of the strain rate, this type of architecture is categorized as rate-dependent. The ANN models are investigated under real-time simulation conditions. In this scenario, the ANN models performed poorly. The results indicate that a richer dataset is required for the ANN models to correctly predict a varying strain rate input.

Finally, the main contribution of this research to the field of soft robotic applications for human assistance is the development and assessment of three novel modelling tools. During this process, an improved way to implemented the Piecewise linearisation method is proposed. Moreover, this research also provided a novel approach to adequate select the actuator technology when designing an assistive device, using the kinetic and kinematic parameters found in clinical studies. Furthermore, the research performed on series-viscoelastic actuators yielded a novel concept design for a clamping mechanisms which has the potential to reduce research and development times when studying soft materials, due its modular design.

\section{Future Work}

The findings of this research have allowed the identification of the following issues still needed to be addressed:

\begin{itemize}
    \item Dynamic mechanical analysis can be used to characterize the soft materials under a wide range of frequencies to create a more comprehensive dataset to train ANN models.
    \item Further investigation on the concept of series and parallel elasticity in combination with soft materials.
    \item Validation of the ANN model when being deployed as part of a control system must be investigated.
    \item Further characterization of the viscoelasticity of soft materials could be done by extracting their hysteresis, creep and cyclic behaviour.
    \item Different input combinations for the ANN model could be explored, mainly the ones which do not have the strain rate of the material as input.
    \item Alternative types of ANNs could be investigated, such as Radial Basis Neural Networks and Recurrent Artificial Neural Networks.
    \item Alternative techniques, such as synthetic data creation, can be investigated to create a richer dataset without having to perform more experimental tests.
    \item Further investigation on the implementation of the concepts of co-contraction and variable recruitment could be performed for soft robotics applications.
    \item Further investigation on the field of the human skeletal system functionality, such as muscle redundancy, which is a compatible approach for cable-driven actuators.
\end{itemize}